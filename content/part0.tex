\chapter{Introduction}

\lettrine{W}{e} start by looking at examples which demonstrate some of the motives behind studying analysis in general.
%
\begin{example*}[Series]
  The geometric series
  \(S = 1 + \frac{1}{2} + \frac{1}{4} + \frac{1}{8} + \frac{1}{16} + \cdots\)
  can be summed by the following simple simple trick.
  Multiplying by \(2\) we obtain that
  \[
    2S = 2 + 1 + \frac{1}{2} + \frac{1}{4} + \frac{1}{8} + \frac{1}{16} + \cdots = 2+S
  \]
  and so \(S=2\).
  If we try to do the same to the sum
  \(T = 1 + 2 + 4 + 8 + 16 + \cdots\)
  we get the nonsensical answer
  \[
    2T = 2 + 4 + 8 + 16 + \cdots = T -1
  \]
  and so \(T = -1\).
  %
  Why should we trust the argument in the first case and not in the second?
\end{example*}


\begin{example*}[Interchanging sums]
  If we consider any matrix of numbers, for example,
  \[
    \begin{pmatrix}
      1 & 2 & 3 \\
      4 & 5 & 6 \\
      7 & 8 & 9
    \end{pmatrix}
  \]
  we can sum first the rows \(6 + 15 + 24 = 45\) or first the columns \(12 + 15 + 18 = 45\) to obtain the total sum of all numbers.
  This is the rule
  \[
    \sum_{j=1}^{m} \sum_{k=1}^{n} a_{jk} = \sum_{k=1}^{n} \sum_{j=1}^{m}  a_{jk}.
  \]
  We would like to believe that also \(\sum_{j=1}^{\infty} \sum_{k=1}^{\infty} a_{jk} = \sum_{k=1}^{\infty} \sum_{j=1}^{\infty}  a_{jk}\).
  However this doesn't work for the following matrix:
  \[
    \begin{pmatrix}
      1      & 0      & 0      & \cdots \\
      -1     & 1      & 0      & \cdots \\
      0      & -1     & 1      & \cdots \\
      \vdots & \vdots & \vdots & \ddots
    \end{pmatrix}
  \]
  %
  We often want to swap the order of summing (or integrating) and often need to consider infinite sums (or integrals).
  When can we do this and can't we?
\end{example*}

\begin{example*}[Interchanging integrals]
  Let's try to integrate \(e^{-xy} - xye^{-xy}\) with respect to both \(x\) and \(y\).
  We would like to believe that
  \[
    \int_{0}^{\infty} \int_{0}^{1} (e^{-xy} - xye^{-xy}) \ dy \ dx
    \overset{\text{\large\color{blue} ?}}{=} \int_{0}^{1} \int_{0}^{\infty}  (e^{-xy} - xye^{-xy}) \ dx \ dy.
  \]
  Since
  \( \int_{0}^{1} (e^{-xy} - xye^{-xy}) \ dy = \left[ ye^{-xy} \right]_{y=0}^{1} = e^{-x}\),
  the left-hand side is
  \( \int_{0}^{\infty} e^{-x} \ dx = \left[ -e^{-x} \right]_{0}^{\infty} = 1 \).
  However, since
  \( \int_{0}^{\infty}  (e^{-xy} - xye^{-xy}) \ dx = \left[ xe^{-xy} \right]_{x=0}^{\infty} = 0\),
  the right-hand side is \(\int_{0}^{1} 0 \ dx = 0\).
  So how do we know when to trust the interchange of intervals?
\end{example*}


\begin{example*}[interchanging limits]
  We could easily believe that
  \[
    \lim_{x\to 0}\lim_{y\to 0} \frac{x^2}{x^2 + y^2}
    \overset{\text{\large\color{blue} ?}}{=}
    \lim_{y\to 0}\lim_{x\to 0} \frac{x^2}{x^2 + y^2}.
  \]
  However \(\lim_{y\to 0} \frac{x^2}{x^2 + y^2} = \frac{x^2}{x^2 + 0} = 1 \) and so the left-hand side is \(1\)
  whereas \(\lim_{x\to 0} \frac{x^2}{x^2 + y^2} = \frac{0}{0 + y^2} = 0\) so the right-hand side is \(0\).
  This example shows that the interchange of limits is untrustworthy. Under what circumstances is it legitimate?
\end{example*}

We need to be rigorous in our logic otherwise, as we have seen in these examples, the conclusions can be erroneous and the difficulties are often subtle.

\subsection*{Curves of constant width}
%
\begin{figure}[htb]
  \centering
  \includesvg[width=0.6\textwidth]{content/graphics/reuleaux.svg}
  \caption{The Reuleaux triangle is a curve of constant width.}
  \label{fig:reuleaux}
\end{figure}
%
The above examples are calculus based but it is worthwhile to consider a real world application of the rigour and reasoning we aspire to.
Suppose we are organising the production facilities which manufacture a component that is round (maybe a rocket body, maybe a tube, etc.).
\begin{samepage}
  As part of the production it is important to have a procedure which guarantees that the fabrication is done to the correct tolerance.
  The idea proposed is:
  \begin{quotation}
    ``We measure the width from all angles to confirm that the manufactured component is correct.''
  \end{quotation}
\end{samepage}
Two-dimensional problem in the sense we assume that the object is a closed curve in \(\bR^2\).
For a given angle we define the width of this curve to be the smallest distance between two parallel lines which touch the curve in a single point but never cross it (one each side of the curve).
We say that the curve has constant width if this width is equal from every direction.
This is just what we would check using calipers on a part and rotating.
The following statement is intuitive and true.
\begin{theorem*}
  A circle has constant width.
\end{theorem*}
\noindent
However the converse is not true, indeed the following is true.
\begin{theorem*}
  There exist constant width curves which are not circles.
\end{theorem*}
\noindent
This can be proved by constructing many such curves, for example the \href{https://en.wikipedia.org/wiki/Reuleaux_triangle}{Reuleaux triangle}. Indeed there are such curves which look similar to regular polygons but still have constant width.


% \footnotetext{
%   Source of Figure~\ref{fig:reuleaux}: 
%   \url{https://commons.wikimedia.org/wiki/File:Reuleaux_supporting_lines.svg}
%   }



\subsection*{MA2 versus MA1}

Much of what we do in this course builds on ideas established in Mathematical Analysis 1.
In particular many of the ideas are extended to the higher dimensional setting.

\begin{center}
  \begin{tabular}{r | l}
    \textbf{Mathematical Analysis 1}
     &
    \textbf{Mathematical Analysis 2}  \\
    \hline
    Sequences \& series of numbers
     &
    Sequences \& series  of functions \\
    \(a_1, a_2, a_3,\ldots \)
     &
    \(f_1(x), f_2(x), f_3(x),\ldots \)
    \\
    \(\sum_{j=0}^{\infty} a_j\)
     &
    \(\sum_{j=0}^{\infty} f_j(x)\)
    \\
    \hline
    (Functions) \(f:\bR \to \bR\)
     &
    \(f:\bR^n \to \bR\) (Scalar fields)
    \\
     &
    \(\mathbf{f}:\bR^n \to \bR^n\) (Vector fields)
    \\
     &
    \(\boldsymbol{\alpha}:\bR \to \bR^n\) (Paths)
    \\
    \hline
    (Derivative) \( f'(x) = \frac{df}{dx}(x)\)
     &
    \( \frac{\partial f}{\partial x_j}(x_1,\ldots,x_n)\) (Partial derivatives)
    \\
     &
    \(\nabla f\) (Gradient)
    \\
     &
    \(D_v f\) (Directional derivative)
    \\
     &
    \(\boldsymbol{\alpha}'\) (Derivative of path)
    \\
     &
    \(Df\) (Jacobian matrix)
    \\
    \hline
    (Extrema) \(\sup_{x\in \bR} f(x)\)
     &
    \(\sup_{x\in \bR^n} f(x)\) (Extrema)
    \\
     &
    Lagrange multiplier method
    \\
    \hline
    Integral \(\int_{a}^{b} f(x) \ dx\)
     &
    Multiple integral
    \\
     &
    Line integral
    \\
     &
    Surface integral
  \end{tabular}
\end{center}


\subsection*{Suggested further reading}

\begin{itemize}
  \item "Analysis 1" by Terence Tao.
        (Particularly \S 1.2 ``Why Analysis?'' and Appendix A ``The basics of mathematical logic'').
\end{itemize}