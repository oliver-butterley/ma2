% !TeX root = ../main.tex
\chapter{Curves \& line integrals}

\lettrine{C}{urves} have played a part in earlier parts of the course and now we turn our attention to precisely what we mean by this notion.
Up until now we relied more on an intuition, an idea of some type of 1D subset of higher dimensional space.
We will also define how we can integrate scalar and vector fields along these curves.
These types of integrals have a natural and important physical relevance.
We will then study some of the properties of these integrals.
To start let's recall a random selection of curves we have already seen:

\begin{center}
    \begin{tabular}{l l}
        Circle           & \(x^2+y^2 = 4\)                               \\
        Semi-circle      & \(x^2+y^2 = 4\), \(x\geq 0\)                  \\
        Ellipse          & \({(\frac{x}{2})}^2 + {(\frac{y}{3})}^2 = 4\) \\
        Line             & \(y=5x+2\)                                    \\
        Line (in 3D)     & \(x+2y+3z=0\), \(x=4y\)                       \\
        Parabola (in 3D) & \(y=x^2\), \(z=x\)
    \end{tabular}
\end{center}

\noindent
In the above list the curves are written in a way where we are describing a set of points using certain constraint or constraints. In some cases in \emph{implicit} form, in some cases in \emph{explicit} form.
For example, for the circle we formally mean the set  \(\{(x,y):x^2+y^2 = 4\}\).
We have the idea that the curves should be sets which are single connected pieces and we vaguely have an idea that we need curves that are sufficiently smooth.
To proceed we need a precise definition of the 1D objects we can work with.
As part of the definition we force a structure which really allows us to work with these objects in a useful way.

\section{Curves, paths \& line integrals}
Let \(\aalpha : [a,b] \to \bR^n\) be continuous.
In components we write \(\aalpha(t) = (\alpha_1(t),\ldots,\alpha_n(t))\).
We say that \(\aalpha(t)\) is \emph{continuously differentiable} if each component \(\alpha_k(t)\) is differentiable on \([a,b]\) and \(\alpha_k'(t)\) is continuous (Definition~\ref{def:differentiable}).
We say that \(\aalpha(t)\) is \emph{piecewise continuously differentiable} if \([a,b] = [a,c_1]\cup[c_1,c_2] \cup \cdots \cup [c_l,b]\) and \(\aalpha(t)\) is \emph{continuously differentiable} on each of these intervals.

\begin{definition}%
    \label{def:path}
    If \(\aalpha: [a,b] \to \bR^n\) is piecewise continuously differentiable then we call it a \emph{path}.
\end{definition}

Note that different functions can trace out the \emph{same} curve in different ways.
Also note that a path has an inherent direction.
We say that this is a \emph{parametric representation} of a given curve.
We already saw examples of paths in Figure~\ref{fig:spiral} and Figure~\ref{fig:particle-circle}.
A few examples of paths are as follows.

\begin{center}
    \begin{tabular}{l l}
        \(\aalpha(t)= (t,t)\),              & \(t\in[0,1]\)                           \\
        \(\aalpha(t) = (\cos t, \sin t)\),  & \(t\in[0,2\pi]\)                        \\
        \(\aalpha(t) = (\cos t, \sin t)\),  & \(t\in [-\frac{\pi}{2},\frac{\pi}{2}]\) \\
        \(\aalpha(t) = (\cos t, -\sin t)\), & \(t\in[0,2\pi]\)                        \\
        \(\aalpha(t)= (t,t,t)\),            & \(t\in[0,1]\)                           \\
        \(\aalpha(t)=(\cos t, \sin t, t)\), & \(t\in [-10,10]\)
    \end{tabular}
\end{center}
\noindent
Observe how some of these parts represent the same circle, perhaps traversed in a different direction.

Let \(\aalpha(t)\) be a (piecewise continuously differentiable) path on \([a,b]\) and
let \(\ff: \bR^n\to\bR^n\) be a continuous vector field.
Recall that we consider \(\aalpha'(t)\) and \(\ff(\xx)\) as \(n\)-vectors.
I.e., in the case \(n=2\), then
\(\aalpha'(t) = \left(
\begin{smallmatrix}
        \alpha_1'(t) \\ \alpha_2'(t)
    \end{smallmatrix}
\right)\)
and \(\ff(\xx) = \left(
\begin{smallmatrix}
        f_1(\xx) \\ f_2(\xx)
    \end{smallmatrix}
\right)\).

\begin{definition}[line integral]%
    \label{def:line-integral}
    The \emph{line integral} of the vector field \(\ff\) along the path \(\aalpha\) is defined as
    \[
        \int \ff \cdot d\aalpha = \int_{a}^{b} \ff(\aalpha(t)) \cdot \aalpha'(t) \ dt.
    \]
\end{definition}


Sometimes the same integral is written as \(\int_C \ff \cdot d\aalpha  \) to emphasize that the integral is along the curve \(C\).
Alternatively the integral is sometimes written as \(\int f_1 \ d\alpha_1 + \cdots + f_n \ d\alpha_n\) or \(\int f_1 \ dx_1 + \cdots + f_n \ dx_n\).
This latter notation highlights the vector nature of the quantity.

\begin{example*}
    Consider the vector field \(\ff(x,y) = \left(\begin{smallmatrix}
        \sqrt y \\ x^3 + y
    \end{smallmatrix}\right)\)
    and the path
    \(\aalpha(t)= (t^2,t^3)\) for \(t \in (0,1)\).
    Evaluate \(\int \ff \cdot d\aalpha\).
\end{example*}
\begin{solution}
    We start by calculating
    \[
        \aalpha'(t) = \begin{pmatrix}
            2t \\ 3t^2
        \end{pmatrix},\\
        \quad
        \ff(\aalpha(t)) = \begin{pmatrix}
            t^{\frac{3}{2}} \\ t^6 + t^3
        \end{pmatrix}.
    \]
    This means that \(\ff(\aalpha(t)) \cdot  \aalpha'(t) =    2 t^{\frac{5}{2}} + 3t^8 + 3t^5\) and so
    \[
        \displaystyle\int \ff \cdot d\aalpha = \displaystyle\int_{0}^{1} (2 t^{\frac{5}{2}} + 3t^8 + 3t^5 ) \ dt = \frac{59}{42}. \qedhere
    \]
\end{solution}

\section{Basic properties of the line integral}

Having defined the line integral, the next step is to clarify its behaviour, in particular the following key properties.

\noindent
\textbf{Linearity:}
Suppose \(\ff\), \(\gg\) are vector fields and \(\aalpha(t)\) is a path. For any \(c,d\in \bR\), then
\[
    \int (c\ff + d\gg) \cdot d\aalpha =  c \int \ff  \cdot d\aalpha +  d \int \gg \cdot d\aalpha.
\]

\noindent
\textbf{Joining / splitting paths:}
Suppose \(\ff\) is a vector field and that
\[
    \aalpha(t) = \begin{cases}
        \aalpha_1(t) & t\in [a,c] \\
        \aalpha_2(t) & t\in [c,b]
    \end{cases}
\]
is a path.
Then
\[
    \int \ff  \cdot d\aalpha = \int \ff  \cdot d\aalpha_1  + \int \ff  \cdot d\aalpha_2.
\]
Alternatively, if we write \(C\), \(C_1\), \(C_2\) for the corresponding curves, then
\[
    \int_{C} \ff  \cdot d\aalpha = \int_{C_1} \ff  \cdot d\aalpha + \int_{C_2} \ff  \cdot d\aalpha.
\]

As already mentioned, for a given curve there are many different choices of parametrization.
For example, consider the curve \(C = \{(x,y) : x^2 + y^2 = 1, y\geq 0\}\).
This is a semi-circle and two possible parametrizations are
\(\aalpha(t) = (-t, \sqrt{1-t^2})\), \(t\in [-1,1]\)
and
\(\bbeta(t) = (\cos t, \sin t)\), \(t\in [0,\pi]\).
These are just two possibilities among many possible choices.
For a given curve, to what extent does the line integral depend on the choice of parametrization?

\begin{definition}[equivalent paths]
    We say that two paths \(\aalpha(t)\) and \(\bbeta(t)\) are \emph{equivalent} if there exists a continuously differentiable function \(u : [c,d] \to [a,b] \) such that \(\aalpha(u(t)) = \bbeta(t)\).

    Furthermore, we say that \(\aalpha(t)\) and \(\bbeta(t)\) are
    \begin{itemize}
        \item \emph{in the same direction} if \(u(c)=a\) and \(u(d)=b\),
        \item \emph{in the opposite direction} if \(u(c)=b\) and \(u(d)=a\).
    \end{itemize}
\end{definition}

With this terminology we can precisely describe the dependence of the integral on the choice of parametrization.

\begin{theorem}[change of parametrization]%
    \label{thm:change-param}
    Let \(\ff\) be a continuous vector field and let \(\aalpha\), \(\bbeta\) be equivalent paths.
    Then
    \[
        \int \ff \cdot d\aalpha =
        \begin{cases}
            \int \ff \cdot d\bbeta   & \text{if the paths are in the same direction},     \\
            - \int \ff \cdot d\bbeta & \text{if the paths are in the opposite direction}.
        \end{cases}
    \]
\end{theorem}

\begin{proof}
    Suppose that the paths are continuously differentiable path, decomposing if required.
    Since \(\aalpha(u(t)) = \bbeta(t)\) the chain rule implies that
    \( \bbeta'(t) = \aalpha'(u(t)) \ u'(t)\).
    In particular
    \[
        \displaystyle \int \ff \cdot  d\bbeta = \displaystyle \int_c^d \ff(\bbeta(t)) \cdot \bbeta'(t) \ dt =  \displaystyle \int_c^d \ff(\aalpha(u(t))) \cdot \aalpha'(u(t)) \ u'(t) \ dt.
    \]
    Changing variables, with a minus sign if path is opposite direction, completes the proof.
\end{proof}


\section{Applications (gradients, work, length)}

Let \(h(x,y)\) be a scalar field in \(\bR^2\)
and recall that the gradient \(\nabla h(x,y)\) is a vector field.
Let \(\aalpha(t)\), \(t\in [0,1]\) be a path.
Now let \(g(t) = h(\aalpha(t))\), consider the derivative
\(g'(t) = \nabla h(\aalpha(t))\cdot \aalpha'(t)\)
and evaluate the line integral
\[
    \begin{aligned}
        \int \nabla h \cdot d\aalpha
         & = \int_0^1 \nabla h(\aalpha(t)) \cdot \alpha'(t)\ dt \\
         & = \int_0^1 g(t) \ dt
        = g(1) - g(0)
        = h(\aalpha(1)) - h(\aalpha(0)).
    \end{aligned}
\]
This equality has the following intuitive interpretation if we suppose for a moment that \(h\) denotes altitude.
In this case the line integral is the sum of all the infinitesimal altitude changes and equals the total change in altitude.

As a first example of work in physics let's consider gravity.
The gravitational field on earth is \(\ff(x,y,z) = \left(\begin{smallmatrix}
        0 \\ 0 \\ mg
    \end{smallmatrix}\right)\).
If we move a particle from \(\aa=(a_1,a_2,a_3)\) to \(\bb=(b_1,b_2,b_3)\) along the path \(\aalpha(t)\), \(t\in [0,1]\)
then the work done is defined as \(\int \ff \cdot d\aalpha\).
We calculate that
\[
    \begin{aligned}
        \int \ff \cdot d\aalpha
         & = \int_0^1 \ff(\aalpha(t)) \cdot \aalpha'(t) \ dt
        = \int_0^1 mg \ \alpha_3'(t) \ dt                          \\
         & = mg \ {\left[ \alpha_3(t)\right]}_0^1 = mg(b_3 - a_3).
    \end{aligned}
\]
This coincides we what we know, work done depends only on the change in height.

As a second example of work in physics let's consider a particle moving in a force field.
Let \(\ff\) be the force field and let \(\xx(t)\) be the position at time \(t\) of a particle moving in the field.
Let \(\vv(t) = \xx'(t)\) be the velocity at time \(t\) of the particle and define kinetic energy as \(\frac{m}{2} \norm{\vv(t)}^2\).
According to Newton's law
\(\ff(\xx(t)) = m\xx''(t) = m \vv'(t)\)
and so the work done is
\[
    \begin{aligned}
        \int \ff \cdot d\xx
         & = \int_0^1 \ff(\xx(t)) \cdot \vv(t) \ dt
        = \int_0^1 m\vv'(t) \cdot \vv(t) \ dt                                   \\
         & = \int_0^1 \tfrac{d}{dt} \left( \tfrac{m}{2} \norm{\vv(t)}^2 \right)
        = \left(  \tfrac{m}{2} \norm{\vv(1)}^2  -  \tfrac{m}{2} \norm{\vv(0)}^2   \right)
    \end{aligned}
\]
In this case we see, as expected, the work done on the particle moving in the force field is equal to the change in kinetic energy.

We also use the line integral to define the \emph{length of a curve} in a meaningful way.
Let \(\aalpha(t)\), \(t\in [a,b]\) be a path.
\begin{definition}[length of a curve]
    The length of the piece of the curve between \(\aalpha(a)\) and \(\aalpha(t)\) is defined as
    \[
        s(t) = \int_a^t \norm{\aalpha'(u)} \ du.
    \]
\end{definition}

Observe that \(s'(t) = \norm{\aalpha'(t)} \) and also that the length of the curve, as we would expect, doesn't depend on the choice of parametrization of that curve.
We also consider the following weighted version of the above definition.
If the path represents a wire and the wire has density \(\varphi(\aalpha(t))\) at the point \(\aalpha(t)\) then the mass of the wire is defined as
\[
    M = \displaystyle\int \varphi(\aalpha(t)) \ s'(t) \ dt.
\]

\section{The second fundamental theorem of calculus}

Recall that, if \(\varphi:\bR \to \bR\) is differentiable then
\(\int_a^b \varphi'(t) \ dt = \varphi(b) - \varphi(a)\).
The analog for line integrals is the following.

\begin{theorem}[2\textsuperscript{nd} fundamental theorem in \(\bR^n\)]
    Suppose that \(\varphi\) is a continuously differentiable scalar field on \(S \subset \bR^n\)
    and suppose that \(\aalpha(t)\), \(t\in[a,b]\) is a path in \(S\).
    Let \(\aa = \aalpha(a)\),  \(\bb = \aalpha(b)\).
    Then
    \[
        \int \nabla \varphi \cdot d\aalpha = \varphi(\bb) - \varphi(\aa).
    \]
\end{theorem}

\begin{proof}
    Suppose that \(\aalpha(t)\) is continuously differentiable.
    By the chain rule \(\frac{d}{dt} \varphi(\aalpha(t)) = \nabla \varphi(\aalpha(t))\cdot \aalpha'(t)\).
    Consequently
    \[
        \int \nabla \varphi \cdot d\aalpha
        = \int_0^1 \nabla \varphi(\aalpha(t)) \cdot \aalpha'(t)\ dt
        = \int_0^1 \tfrac{d}{dt} \varphi(\aalpha(t)) \ dt.
    \]
    By the 2\textsuperscript{nd} fundamental theorem in \(\bR\) we know that
    \(\int_0^1 \tfrac{d}{dt} \varphi(\aalpha(t)) \ dt = \varphi(\aalpha(b)) - \varphi(\aalpha(a))\).
\end{proof}

\begin{example}[potential energy]
    Our earth has mass \(M\) with centre at \((0,0,0)\).
    Suppose that there is a small particle close to earth which has mass \(m\).
    The force field of gravitation and potential energy are, respectively,
    \[
        \ff(\xx) = \frac{-GmM}{\norm{\xx}^3}\xx,
        \quad
        \varphi(\xx) = \frac{GmM}{\norm{\xx}}.
    \]
    We can calculate \(\nabla \varphi(\xx)\) and see that it is equal to \(\ff(\xx)\).
\end{example}


\section{The first fundamental theorem of calculus}

First we need to consider a basic topological property of sets.
In particular we want to avoid the possibility of the set being several disconnected pieces, in other words we want to guarantee that we can get from one point to another in the set in a way without every leaving the set (see Figure~\ref{fig:connected}).

\begin{definition}[connected]
    The set \(S\subset \bR^n\) is said to be \emph{connected} if, for every pair of points \(\aa,\bb\in S\), there exists a path \(\aalpha(t), t\in[a,b]\) such that
    \begin{itemize}
        \item \(\aalpha(t)\in S\) for every \( t\in[a,b]\),
        \item \(\aalpha(a)=\aa\) and \(\aalpha(b)=\bb\).
    \end{itemize}
\end{definition}

\noindent
Sometimes this property is called ``path connected'' to distinguish between different notions.

\begin{figure}[htbp]
    \begin{center}
        \includegraphics{connected.pdf}
        \caption{A connected set.}%
        \label{fig:connected}
    \end{center}
\end{figure}

Recall that, if \(f:\bR \to \bR\) is continuous and \(\varphi(x) := \int_a^x f(t) \ dt\) then \(\varphi'(x) = f(x)\).

\begin{theorem}[1\textsuperscript{st} fundamental theorem in \(\bR^n\)]
    Let \(\ff\) be a continuous vector field on a connected set \(S \subset \bR^n\).
    Suppose that, for \(\xx,\aa\in S\), the line integral \(\int \ff \cdot d\aalpha\) is equal for every path \(\aalpha\) such that \(\aalpha(a)=\aa\), \(\aalpha(b)=\xx\).
    Fix \(\aa\in S\) and define \(\varphi(\xx)= \int \ff \cdot d\alpha\).
    Then \(\varphi\) is continuously differentiable and \(\nabla \varphi = \ff\).
\end{theorem}

\begin{proof}[Sketch of proof]

    As before let \(\ee_1 = \left(\begin{smallmatrix}
            1 \\ 0 \\ 0
        \end{smallmatrix}\right) \),
    \(\ee_2 = \left(\begin{smallmatrix}
            0 \\ 1 \\ 0
        \end{smallmatrix}\right) \),
    \(\ee_3 = \left(\begin{smallmatrix}
            0 \\ 0 \\ 1
        \end{smallmatrix}\right) \).
    Observe that, if we define the paths \(\bbeta_k(t) = \xx + t \ee_k\), \(t\in [0,h]\), then
    \[
        \varphi(\xx + h \ee_k) - \varphi(\xx) = \int \ff \cdot d\bbeta_k.
    \]
    Moreover \(\bbeta'_k(t) = \ee_k\).
    Consequently
    \[
        \begin{aligned}
            \frac{\partial \varphi}{\partial x_k}(\xx)
             & =  \displaystyle\lim_{h\to 0} \frac{1}{h}( \varphi(\xx + h \ee_k) - \varphi(\xx))                \\
             & = \displaystyle \lim_{h\to 0} \frac{1}{h} \int_0^h \ff(\bbeta_k(t)) \cdot \ee_k \ dt = f_k(\xx).
        \end{aligned}
    \]
    In other words, we have shown that \(\nabla \varphi (\xx) =  \ff(\xx)\).
\end{proof}

\begin{definition}[closed path]
    We say a path \(\aalpha(t)\), \(t\in [a,b]\) is \emph{closed} if \(\aalpha(a) = \aalpha(b)\).
\end{definition}

Observe that, if \(\aalpha(t)\), \(t\in[a,b]\) is a closed path then we can divided it into two paths: Let \(c\in[a,b]\) and consider the two paths \(\aalpha(t)\), \(t\in[a,c]\) and  \(\aalpha(t)\), \(t\in[c,b]\).
On the other hand, suppose \(\aalpha(t)\), \(t\in [a,b]\) and  \(\bbeta(t)\), \(t\in [c,d]\) are two path starting at \(\aa\) and finishing at \(\bb\). The these can be combined to define a closed path (by following one backward).

\begin{definition}[conservative vector field]
    A vector field \(\ff\), continuous on \(S \subset \bR^n\) is \emph{conservative} if there exists a scalar field \(\varphi\) such that, on \(S\), \[\ff = \nabla \varphi.\]
\end{definition}

Note that some authors call such a vector field a \emph{gradient} (i.e., the vector field is the gradient of some scalar).
If \(\ff = \nabla \varphi\) then the scalar field \(\varphi\) is called the \emph{potential} (associated to \(\ff\)).
Observe that that the potential is not unique,
\(\nabla \varphi = \nabla(\varphi + C)\) for any constant \(C \in \bR\).


\begin{theorem}[conservative vector fields]
    The following are equivalent for a vector field \(\ff\):
    \begin{enumerate}[label=\textnormal{(\roman*)}]
        \item There exists \(\varphi\) such that \(\ff = \nabla \varphi\),
        \item \(\int \ff \cdot d\aalpha\) does not depend on \(\aalpha\), as long as \(\aalpha(a)=\aa\), \(\aalpha(b)=\bb\),
        \item \(\int \ff \cdot d\aalpha = 0\) for any closed path \(\aalpha\).
    \end{enumerate}
\end{theorem}

\begin{proof}
    In the previous theorems (the two fundamental theorems) we proved that (i) is equivalent to (ii).

    Now we prove that (ii) implies (iii):
    Let \(\aalpha(t)\) be a closed path and let \(\bbeta(t)\) be the same path in the opposite direction. Observe that \(\int \ff \cdot d\aalpha = - \int \ff \cdot d \bbeta\) but that \(\int \ff \cdot d \aalpha = \int \ff \cdot d \bbeta\) and so \(\int \ff \cdot d \aalpha = 0\).

    It remains to prove that (iii) implies (ii): The two paths between \(\aa\) and \(\bb\) can be combined (with a minus sign) to give a closed path.
\end{proof}

\begin{theorem}%
    \label{thm:mixed-partials}
    Suppose that \(\ff\) is a continuously differential vector field.
    If \(\ff = \nabla \varphi\) for some scalar field \(\varphi\) then, for each \(l,k\),
    \[
        \frac{\partial f_l}{\partial x_k} = \frac{\partial f_k}{\partial x_l}.
    \]
\end{theorem}

\begin{proof}
    By assumption the second order partial derivatives exist and so
    \[
        \tfrac{\partial f_l}{\partial x_k}
        = \tfrac{\partial^2 \varphi}{\partial x_k \partial x_l}
        = \tfrac{\partial^2 \varphi}{\partial x_l \partial x_k}
        = \tfrac{\partial f_k}{\partial x_l}. \qedhere
    \]
\end{proof}


\begin{example}
    Consider the vector field
    \[
        \ff(x,y) = \left(\begin{smallmatrix}
                -y {(x^2 + y^2)}^{-1} \\ x {(x^2 + y^2)}^{-1}
            \end{smallmatrix}\right)
    \]
    on \(S = \bR^2 \setminus (0,0)\).
    Calculating we verify that  \(  \tfrac{\partial f_1}{\partial y} = \tfrac{\partial f_2}{\partial x}\) on \(S\).
    We now evaluate the line integral \(\int \ff \cdot d\aalpha\) where \(\aalpha(t) = (a \cos t, a \sin t)\), \(t\in [0,2\pi]\).
    We calculate that
    \(\aalpha'(t) = \left(\begin{smallmatrix}
            -a \sin t \\ a \cos t
        \end{smallmatrix}\right)\) and \(\ff(\aalpha(t)) = \frac{1}{a^2} \left(\begin{smallmatrix}
            -a \sin t \\ a \cos t
        \end{smallmatrix}\right)   \).
    This means that
    \[
        \int \ff \cdot d\aalpha = \int_{0}^{2\pi} ( \sin^2 t + \cos^2 t )\ dt = 2\pi.
    \]
\end{example}
Observe that in the above example \(S\) is somehow not a ``nice'' set because of the ``hole'' in the middle.
Moreover, observe that the line integral is the same for any circle, independent of the radius.

Theorem~\ref{thm:mixed-partials} isn't really useful in showing that a vector field is conservative because it is possible for the mixed partial derivatives to all be equal but still the field fail to be conservative.
On the other hand, if a pair of mixed derivatives is not equal then \(\ff\) is \emph{not} conservative and so it is useful for proving the negative.
Later in this chapter we will return to this topic.

\section{Potential functions and conservative vector fields}

We now turn our attention to the following question:
Suppose we are given a vector field \(\ff\) and we know that \(\ff = \nabla \varphi\) for some \(\varphi\).
How can we find \(\varphi\)?
For this we consider two methods in the following paragraphs.
%
\begin{figure}[tbhp]
    \centering
    \includegraphics{two-paths.pdf}
    \caption{The paths \(\aalpha_1\) and \(\aalpha_2\).}
\end{figure}
%
First we describe the method which we call \emph{constructing a potential by line integral}.
Suppose that \(\ff\) is a conservative vector field on the rectangle \([a_1,b_1]\times [a_2,b_2]\).
We define \(\varphi(\xx)\) as the line integral \(\int \ff \cdot d\aalpha\) where \(\aalpha\) is a path between \(\aa=(a_1,a_2)\) and \(\xx\).
For any \(\xx = (x_1,x_2) \in \bR^2\) consider the two paths:
\begin{itemize}
    \item[] \(\aalpha_1(t) = (t,a_2)\), \(t\in [a_1,x_1]\),
    \item[]  \(\aalpha_2(t) = (x_1,t)\),  \(t\in [a_2,x_2]\).
\end{itemize}
Let \(\aalpha(t)\) denote the concatenation of the two paths.
We calculate that
\[
    \int \ff \cdot d\aalpha = \int_{a_1}^{x_1} \ff(\aalpha_1(t))\cdot \aalpha_1'(t) \ dt +  \int_{a_2}^{x_2} \ff(\aalpha_2(t))\cdot \aalpha_2'(t) \ dt.
\]
This means that \(\varphi(\xx)  = \int_{a_1}^{x_1} f_1(t,a_2)\ dt + \int_{a_2}^{x_2} f_2(x_1,t)\ dt \).

Now we describe a different method which we describe as \emph{constructing a potential by indefinite integrals}.
Again suppose that \(\ff = \nabla \varphi\) for some scalar field \(\varphi(x,y)\) which we wish to find.
Observe that \(\frac{\partial \varphi}{\partial x} = f_1\) and  \(\frac{\partial \varphi}{\partial y} = f_2\).
This means that
\[
    \int_{a}^{x} f_1(t,y) \ dt + A(y) = \varphi(x,y) =  \int_{b}^{y} f_2(x,t) \ dt + B(x)
\]
where \(A(y)\), \(B(x)\) are constants of integration.
Calculating and comparing we can then obtain a formula for \(\varphi(x,y)\).

\begin{example*}
    Find a potential for \(\ff(x,y) = \left(\begin{smallmatrix}
        e^x y^2 + 1\\ 2e^x y
    \end{smallmatrix}\right)\) on \(\bR^2\).
\end{example*}
\begin{solution}
    We calculate that
    \[
        \begin{aligned}
            \int_{a}^{x} f_1(t,y) \ dt + A(y)
             & =  \ e^x y^2 + x + A(y) = \varphi(x,y), \\
            \int_{b}^{y} f_2(x,t) \ dt + B(x)
             & =  e^x y^2 + B(x) = \varphi(x,y).
        \end{aligned}
    \]
    From this we see that we can choose \(A(y) = 0\) and \(B(x)=x\) to obtain equality of the above quantities.
    Consequently we obtain the potential  \(\varphi(x,y) = e^x y^2 + x\).
\end{solution}

Theorem~\ref{thm:mixed-partials} concerning conservative fields and the mixed partial derivatives was somewhat less than satisfactory since the converse wasn't possible.
In order to get a more satisfactory result we need to look at another topological details of the domain of the vector field.
This concept is somewhat suggested by the methods of constructing potentials which were described above.

\begin{definition}[convex set]%
    \label{def:convex}
    A set \(S\subset \bR^n\) is said to be \emph{convex} if for any \(\xx,\yy\in S\) the segment \(\{t\xx + (1-t)\yy, t\in[0,1]\}\) is contained in \(S\).
\end{definition}

\begin{figure}
    \centering
    \begin{subfigure}[b]{0.5\textwidth}
        \includegraphics{convex.pdf}
        \caption{A convex set.}
    \end{subfigure}%
    \begin{subfigure}[b]{0.5\textwidth}
        \includegraphics{not-convex.pdf}
        \caption{A set which is not convex.}
    \end{subfigure}
    \caption{Convex and non-convex sets.}
\end{figure}

This extra property permits the following sufficient condition for a vector field to be conservative.

\begin{theorem}
    Let\footnote{As usual  \(  f_k(x_1,\ldots,x_n)\) denotes the \(k\)\textsuperscript{th} component of the vector field \(\ff\).} \(\ff\) be a continuously differentiable vector field on a convex region \(S\subset \bR^n\).
    Then \(\ff\) is conservative if and only if
    \[
        \tfrac{\partial f_l}{\partial x_k} = \tfrac{\partial f_k}{\partial x_l},
        \quad \text{for each \(l,k\)}.
    \]
\end{theorem}

\begin{proof}[Sketch of proof]
    We have already proved that \(\ff\) being conservative implies the equality of partial derivatives (Theorem~\ref{thm:mixed-partials}) and therefore we need only assume that \(\partial_g f_l = \partial_l f_k\) and construct a potential.
    Let \(\varphi(\xx) = \int \ff \cdot d\aalpha\) where \(\aalpha(t) = t\xx\), \(t\in[0,1]\).
    Since \(\aalpha'(t) = \xx\), \(\varphi(\xx) = \int_0^1 \ff(t\xx)\cdot \xx \ dt\).
    Also (needs proving)
    \[
        \frac{\partial \varphi}{\partial x_k}(t\xx) = \int_{0}^{1} \left( t \partial_k \ff(t\xx) \cdot \xx + f_k(t\xx) \right) \ dt.
    \]
    This is equal to \(\int_{0}^{1} \left( t \nabla f_k(t\xx) \cdot \xx + f_k(t\xx) \right) \ dt\) because  \(\partial_g f_l = \partial_l f_k\);
    By the chain rule applied to \(g(t) = t \nabla f_k(t\xx) \) this is equal to \(f_k(\xx)\) as required.
\end{proof}

The above gives us a few tools to check if a given vector field is conservative.


\subsubsection*{Application to exact differential equations}

A  differential equation of the form \(p(x,y) + q(x,y) \frac{d y}{d x}=0\) is called an \emph{exact differential equation}.

\begin{theorem}%
    \label{thm:exact-diff-eq}
    If \(\varphi(x,y)\) satisfies
    \(\nabla \varphi(x,y) =   \left(\begin{smallmatrix}
            p(x,y) \\ q(x,y)
        \end{smallmatrix}\right)\)
    then the solution \(y(x)\) of the equation \(p(x,y) = q(x,y) \frac{d y}{d x}\) satisfies \(\varphi(x,y(x))=C\) for some \(C\in \bR\).

    Conversely, if  \(\varphi(x,y(x))=C\)  defines implicitly a function \(y(x)\), then \(y(x)\) is a solution to the equation  \(p(x,y) = q(x,y) \frac{d y}{d x}\).
\end{theorem}

\begin{proof}
    If \(y(x)\) satisfies  \(\varphi(x,y(x))=C\), then by the chain rule and \(\nabla \varphi =   \left(\begin{smallmatrix}
        p \\ q
    \end{smallmatrix}\right)\), \(p(x,y(x)) + y'(x) q(x,y(x)) = 0\).

    Conversely, if \(y(x)\) is a solution, \(\varphi(x,y(x))\) must be constant in \(x\).
\end{proof}

\begin{example}
    Solve \(y^2 + 2xyy' = 0\).
    Let \(p(x,y) = y^2\), \(q(x,y) = 2xy\) and find \(\varphi(x,y) = xy^2\) so \(\nabla \varphi = \left(\begin{smallmatrix}
            p\\ q
        \end{smallmatrix}\right)\).
    Solutions satisfy \(\varphi(x,y(x))= x {y(x)}^2 =C\), i.e., \(y(x) = \sqrt{\frac{C}{x}}\).
\end{example}