% !TeX root = ../main.tex
\chapter{Multiple integrals}

\lettrine{T}{he} extension to higher dimension of differentiation was established first.
We then defined line integrals which are, in a sense, one dimensional integrals which exist in a high dimensional setting.
We now take the next step and define higher dimensional integrals.

\section{Definition of the integral}

First we need to find a definition of integrability and the integral.
Then we will proceed to study the properties of this higher dimensional integral.
Recall that, in the one-dimensional case integration was defined using the following steps:
\begin{enumerate}
    \item Define the integral for step functions,
    \item Define integral for ``integrable functions'',
    \item Show that continuous functions are integrable.
\end{enumerate}
For higher dimensions we follow the same logic.
We will then show that we can evaluate higher dimensional integrals by repeated one-dimensional integration.

\subsection{Partitions \& step functions}

\begin{figure}
    \centering
    \includegraphics{partition.pdf}
    \caption{A partition of a rectangle \(R\).}
\end{figure}

\begin{definition*}[partition]
    Let \(R = [a_1,b_1] \times [a_2,b_2]\) be a rectangle.
    Suppose that \(P_1 = \{x_0,\ldots,x_m\}\) and \(P_2 = \{y_0,\ldots,y_n\}\) such that
    \(a_1 = x_0 < x_2 < \cdots < x_m = b_1\) and \(a_2 = y_0 < y_2 < \cdots < y_n = b_2\).
    \(P= P_1 \times P_2\) is said to be a partition of \(R\).
\end{definition*}

Observe that a partition divides \(R\) into \(nm\) sub-rectangles.
If \(P \subseteq Q\) then we say that \(Q\) is a finer partition than \(P\).
Partitions are constructed in higher dimension, for \(\bR^n\), in an analogous way.
Before defining integration for general functions it is convenient to make the definition for a special class of functions called step functions.

\begin{definition}[step function]
    A function \(f:R\to \bR \) is said to be a \emph{step function} if there is a partition \(P\) of \(R\) such that \(f\) is constant on each sub-rectangle of the partition.
\end{definition}

\begin{figure}
    \centering
    \includegraphics{step-function.pdf}
    \caption{Graph of a step function.}
\end{figure}

Observe that, if \(f\) and \(q\) are step functions and  \(c,d \in \bR\), then \(c f + d g\) is also a step function.
Also note that the area of the sub-rectangle \(Q_{jk}:=[x_{j},x_{j+1}]\times [y_{k},y_{k+1}]\) is equal to \( (x_{j+1}-x_{j})(y_{k+1}-y_{k})\).

\subsection{Integral of a step function}

\begin{definition}[integral of a step function]
    Suppose that \(f\) is a step function with value \(c_{jk}\) on the sub-rectangle \((x_{j},x_{j+1})\times (y_{k},y_{k+1})\).
    Then we define the integral as
    \[
        \iint_{R} f \ dxdy = \sum_{j=0}^{m-1} \sum_{k=0}^{n-1} c_{jk} (x_{j+1}-x_{j})(y_{k+1}-y_{k}).
    \]
\end{definition}

Observe that the value of the integral is independent of the partition, as long as the function is constant on each sub-rectangle.
Also note that, if \(f\) is constant on the partition element \(Q_{jk} = (x_{j},x_{j+1})\times (y_{k},y_{k+1})\), then
\[
    \begin{aligned}
        \iint_{Q_{jk}} f \ dxdy
         & = c_{jk} (x_{j+1}-x_{j})(y_{k+1}-y_{k})                                       \\
         & = \int_{x_j}^{x_{j+1}} \left[\int_{y_k}^{y_{k+1}}f(x,y) \ dy\right] \ dx      \\
         & =  \int_{y_k}^{y_{k+1}} \left[ \int_{x_j}^{x_{j+1}} f(x,y) \ dx \right] \ dy.
    \end{aligned}
\]

\begin{theorem}[basic properties of the integral]
    Let \(f,g\) be step functions.
    \begin{enumerate}
        \item \(\displaystyle \iint_{R} (a f + b g) \ dxdy = a \displaystyle \iint_{R} f \ dxdy + b \displaystyle\iint_{R} g \ dxdy\) for all \(a,b\in \bR\);
        \item \(\displaystyle\iint_{R} f \ dxdy =  \displaystyle\iint_{R_1} f \ dxdy + \displaystyle \iint_{R_2} f \ dxdy\) if \(R\) is divided into \(R_1\) and \(R_2\);
        \item \(\displaystyle\iint_{R} f \ dxdy \leq \displaystyle \iint_{R} g \ dxdy\) if \(f(x,y) \leq g(x,y)\).
    \end{enumerate}
\end{theorem}

\begin{proof}
    All properties follow from the definition by basic calculations.
\end{proof}



\subsection{Definition of integrable}


In order to define integrability we take advantage of ``upper'' and ``lower'' integrals which ``sandwich'' the function we really want to integrate.

\begin{definition}[integrability]
    Let \(R\) be a rectangle and let \(f: R \to \bR\) be a bounded function.
    If there is one and only one number \(I\in \bR\) such that
    \[
        \iint_{R} g(x,y) \ dxdy \leq I \leq \iint_{R} h(x,y) \ dxdy
    \]
    for every pair of step functions \( g, h\) such that, for all \((x,y)\in R\),
    \[
        g(x,y) \leq f(x,y) \leq h(x,y).
    \]
    This number \(I\) is called the integral of \(f\) on \(R\) and is denoted \(\iint_{R} f(x,y) \ dxdy\).
\end{definition}


\subsection{Evaluation of an integral}

Now we have a definition we can rigorously work with integrals but it is essential to also have a way to practically evaluate any given integral.

\begin{theorem*}[evaluating by repeated integration]
    Let \(f\) be a bounded integrable function on  \(R = [a_1,b_1] \times [a_2,b_2]\).
    Suppose that, for every \(y\in [a_2,b_2]\), the integral \( A(y) = \int_{a_1}^{b_1} f(x,y) \ dx \) exists.

    Then \(\int_{a_2}^{b_2} A(y) \ dy\) exists and,
    \[
        \iint_{R} f \ dxdy = \int_{a_2}^{b_2} \left[ \int_{a_1}^{b_1} f(x,y) \ dx  \right] \ dy.
    \]
\end{theorem*}

\begin{proof}
    We start by choosing step functions \(g,h\) such that \(g\leq f \leq h\).
    By assumption \(\int_{a_1}^{b_1} g(x,y) \ dx \leq A(y) \leq \int_{a_1}^{b_1} h(x,y) \ dx\).
    We then observe that \(\int_{a_1}^{b_1} g(x,y) \ dx \) and \(\int_{a_1}^{b_1} h(x,y) \ dx\) are step functions (in \(y\)) and so \(A(y)\) is integrable.
    Moreover,
    \[
        \begin{aligned}
            \int_{a_2}^{b_2} \left[ \int_{a_1}^{b_1} g(x,y) \ dx  \right] \ dy
             & \leq \int_{a_2}^{b_2} \left[ \int_{a_1}^{b_1} f(x,y) \ dx  \right] \ dy  \\
             & \leq \int_{a_2}^{b_2} \left[ \int_{a_1}^{b_1} h(x,y) \ dx  \right] \ dy.
        \end{aligned}
    \]
\end{proof}

The conditions of the above theorem aren't immediately easy to check and so it is convenient to now investigate the integrability of continuous functions.

\begin{theorem}[integral of continuous functions]
    Suppose that \(f\) is a continuous function defined on the rectangle \(R\).
    Then \(f\) is integrable and
    \[
        \iint_{R} f(x,y) \ dxdy
        = \int_{a_2}^{b_2} \left[ \int_{a_1}^{b_1} f(x,y) \ dx  \right] \ dy
        = \int_{a_1}^{b_1} \left[\int_{a_2}^{b_2}  f(x,y) \ dy  \right] \ dx.
    \]
\end{theorem}
\begin{proof}
    Continuity implies boundedness and so upper and lower integrals exist.
    Let \(\epsilon>0\). Exists \(\delta>0\) such that \(\abs{f(\xx)-f(\yy)}\leq \epsilon\) whenever \(\norm{\xx-\yy}\leq \delta\).
    We can choose a partition such  \(\norm{\xx-\yy}\leq \delta\) whenever \(\xx,\yy\) are in the same sub-rectangle \(Q_{jk}\).
    We then define the step functions \(g,h\) s.t. \(g(\xx)=\inf_{Q{jk}} f\),   \(h(\xx)=\sup_{Q{jk}} f\) when \(\xx\in Q_{jk}\).
    To finish the proof we observe that \(\abs{\inf_{Q{jk}} f - \sup_{Q{jk}} f }\leq \epsilon\) and \(\epsilon>0\) can be made arbitrarily small.
\end{proof}

This integral naturally allows us to calculate the volume of a solid.
Let \(f(x,y)\leq z \leq g(x,y)\) be defined on the rectangle \(R \subset \bR^2\) and consider the 3D set defined as
\[
    V = \left\{(x,y,z): (x,y)\in R, f(x,y)\leq z \leq g(x,y) \right\}.
\]
The volume of the set \(V\) is equal to
          \[
              \operatorname{Vol}(V) = \iint_{R} \left(g(x,y) - f(x,y)\right)  \ dxdy.
          \]

\begin{figure}
    \centering
    \includegraphics{volume.pdf}
    \caption{Set enclosed by \(xy\)-plane and \(f(x,y)\).}
\end{figure}


Up until now we have considered step function and continuous functions.
Clearly we can permit some discontinuities and we introduce the following concept to be able to control the functions with discontinuities sufficiently to guarantee that the integrals are well-defined.

\begin{definition}[content zero set]
    A bounded subset \(A\subset \bR^2\) is said to have content zero if, for every \(\epsilon>0\), there exists a finite set of rectangles whose union includes \(A\) and the sum of the areas of the rectangles is not greater than \(\epsilon\).
\end{definition}

Examples of content zero sets include:  Finite set of points; Bounded segment; Continuous path.

\begin{figure}
    \centering
    \includegraphics{content-zero.pdf}
    \caption{The graph of a continuous function has content zero.}
\end{figure}

\begin{theorem}
    Let \(f\) be a bounded function on \(R\) and suppose that the set of discontinuities \(A\subset R\) had content zero. Then the double integral \(\iint_{R}f(x,y) \ dxdy\) exists.
\end{theorem}
\begin{proof}
 Take a cover of \(A\) by rectangles with total area not greater than \(\delta>0\).
  Let \(P\) be a partition of \(R\) which is finer than the cover of \(A\).
     We may assume that \(\abs{\inf_{Q{jk}} f - \sup_{Q{jk}} f }\leq \epsilon\)  on each sub-rectangle of the partition which doesn't contain a discontinuity of \(f\).
     The contribution to the integral of bounding step functions from the cover of \(A\) is bounded by \(\delta \sup \abs{f}\).
\end{proof}

\subsection{Integrals over regions bounded by continuous functions}

\begin{definition}[integral on general regions]
    Suppose \(S\subset R\) and \(f\) is a bounded function on \(S\).
    We extend \(f\) to \(R\) by defining
    \[
        f_R(x,y) := \begin{cases}
            f(x,y) & \text{if \((x,y)\in S\)} \\
            0      & \text{otherwise}.
        \end{cases}
    \]
    We say that \(f\) is integrable if \(f_{R}\) is integrable and define
    \[
        \iint_{S} f(x,y) \ dxdy = \iint_{R} f_{R}(x,y) \ dx dy.
    \]
\end{definition}

{Regions bounded by continuous functions}

\begin{definition}[type 1]
    \(S \subset \bR^2\) is \emph{type 1} if there are continuous functions \(\varphi_1\), \(\varphi_2\) such that
    \[
        S = \left\{(x,y): x \in [a,b], \varphi_1(x) \leq y \leq \varphi_2(x)\right\}.
    \]
\end{definition}
\begin{definition}[type 2]
    \(S \subset \bR^2\) is \emph{type 2} if there are continuous functions \(\varphi_1\), \(\varphi_2\) such that
    \[
        S = \left\{(x,y): y \in [a,b], \varphi_1(y) \leq x \leq \varphi_2(y)\right\}.
    \]
\end{definition}



\begin{figure}
    \centering
    \includegraphics{type-one.pdf}
    \caption{A region of type 1.}
\end{figure}

\begin{theorem}
    Let \(\varphi\) be a continuous function on \([a,b]\).
    Then the graph
    \(\left\{(x,y): x\in [a,b], y=\varphi(x)\right\}\)
    has zero content.
\end{theorem}
\begin{proof}
    \begin{enumerate}
        \item Continuity means that, for every \(\epsilon>0\), there exists \(\delta>0\) such that \(\abs{\varphi(x) - \varphi(y)}\leq \epsilon\) whenever \(\abs{x-y}\leq \delta\),
        \item Take partition of \([a,b]\) into subintervals of length less than \(\delta\),
        \item Using this partition we generate a cover of the graph which has area not greater than \(2\epsilon \abs{b-a}\).
    \end{enumerate}
\end{proof}


{Integrals over regions bounded by continuous functions}

\begin{theorem}
    Let \(S = \left\{(x,y): x \in [a,b], \varphi_1(x) \leq y \leq \varphi_2(x)\right\}\) be a region of type 1 and let \(f\) be a bounded continuous function of \(S\).
    Then \(f\) is integrable on \(S\) and
    \[
        \iint_{S} f(x,y) \ dxdy = \int_{a}^{b} \left[\int_{\varphi_1(x)}^{\varphi_2(x)} f(x,y) \ dy\right] \ dx.
    \]
\end{theorem}

\begin{proof}
    \begin{enumerate}
        \item The set of discontinuity of \(f_{R}\) is the boundary of \(S\) in \(R=[a,b]\times[\tilde a,\tilde b]\) which consists of the graphs of \(\varphi_1\), \(\varphi_2\),
        \item These graphs have zero content as we proved before,
        \item For each \(x\), \(f(x,y)\) is integrable since it has only two discontinuity points,
        \item Additionally \(\int_{\tilde a}^{\tilde b} f_{R}(x,y) \ dy =  \int_{\varphi_1(x)}^{\varphi_2(x)} f(x,y) \ dy \). \qedhere
    \end{enumerate}
\end{proof}

A similar result holds for type 2 regions.

\subsection{Applications of multiple integrals}

% {Area and Volume}

Area of \(S\).
Let    \(S \subset \bR^2\)  be a \emph{type 1} region, i.e.,   \(\varphi_1\), \(\varphi_2\) are continuous functions,
\[
    S := \left\{(x,y): x \in [a,b], \varphi_1(x) \leq y \leq \varphi_2(x)\right\}.
\]
Integrating:
\[
    \iint_{S} \ dx dy = \int_{a}^{b}( \varphi_2(x) - \varphi_1(x)) \ dx.
\]


Volume of \(V\):
Let \(f(x,y) \geq g(x,y)\) be continuous functions on \(S\) and
\[
    V := \left\{ (x,y,z) :  x \in [a,b], \varphi_1(x) \leq y \leq \varphi_2(x), f(x,y) \leq z \leq g(x,y)  \right\}.
\]
Integrating:
\[
    \iiint_{V} \ dxdydz =  \int_{a}^{b}
    \left[ \int_{\varphi_1(x)}^{\varphi_2(x)}   ( f(x,y)  -  g(x,y)) dy \right] \ dx.
\]




{Mass, centre of mass, centroid}


\begin{itemize}
    \item Suppose we have several particles\footnote{In general, mass \(m_k\) at point \(\xx_k\), the centre of mass is point \(\mathbf{X}\) such that \( M\mathbf{X} = \sum_{k} m_k \xx_k\).} each with mass \(m_k\) at point \((x_k,y_k)\).
          \begin{itemize}
              \item Total mass is \(M := \sum_{k} m_k\),
              \item Centre of mass is the point \((p,q)\) such that
                    \[
                        p M = \sum_{k} m_k x_k,
                        \quad
                        q M = \sum_{k} m_k y_k.
                    \]
          \end{itemize}
    \item Suppose a disk has the shape of a region \(S\) and the density of the material is \(f(x,y)\) at point \((x,y)\).
          \begin{itemize}
              \item Total mass is \(M := \iint_{S} f(x,y) \ dxdy\),
              \item Centre of mass is the point \((p,q)\) such that
                    \[
                        p M = \iint_{S} x \ f(x,y) \ dxdy,
                        \quad
                        q M = \iint_{S} y \ f(x,y) \ dxdy.
                    \]
          \end{itemize}
    \item If the density is constant the centre of mass is called the centroid.
\end{itemize}




\section{Green's theorem}


 {Green's theorem}

\begin{theorem}[Green's theorem]
    Let \(C\subset\bR^2\) be a piecewise-smooth simple (no intersections) curve and \(\aalpha\) a path that parametrizes \(C\) in the counter-clockwise direction.
    Let \(S\) be the region enclosed by \(C\).
    Suppose that \(\ff(x,y) = \left(\begin{smallmatrix}
            P(x,y) \\ Q(x,y)
        \end{smallmatrix}\right)\) is a vector field continuously differentiable  on an open set containing \(S\).
    Then
    \[
        \iint_{S} \left(\tfrac{\partial Q}{\partial x} - \tfrac{\partial P}{\partial y}\right) \ dxdy = \int_{C} \ff \cdot d\aalpha.
    \]
\end{theorem}





{}


\begin{proof}[Proof of Green's theorem]
    \begin{enumerate}
        \item Assume that \(S\) is a type 1 region and that \(Q=0\),
        \item Since \(S = \left\{(x,y): x \in [a,b], \varphi_1(x) \leq y \leq \varphi_2(x)\right\}\),
              \[
                  \iint_{S}  \left(\tfrac{\partial Q}{\partial x} - \tfrac{\partial P}{\partial y}\right) \ dxdy =
                  \int_{a}^{b}\left[\int_{\varphi_1(x)}^{\varphi_2(x)}  (- \tfrac{\partial P}{\partial y}) \ dy\right] \ dx
                  = \int_{a}^{b}  (P(x,\varphi_1(x))-P(x,\varphi_2(x)))   dx,
              \]
        \item Choose paths \(\aalpha_1(t) = (t,\varphi_1(t))\), \(\aalpha_2(t) = (a,t)\), \(\aalpha_3(t) = (t,\varphi_2(t))\), \(\aalpha_4(t) = (b,t)\),
        \item \(\int_{C} \ff \cdot d\aalpha = \int \ff \cdot d\aalpha_1 - \int \ff \cdot d\aalpha_3 = \int_{a}^{b} P(t,\varphi_1(t)) \ dt -  \int_{a}^{b} P(t,\varphi_2(t)) \ dt \),
        \item If \(S\) is also type 2 then this works for \(P=0\) and linearity means it works for \(\ff = \left(\begin{smallmatrix}
                  P \\ 0
              \end{smallmatrix}\right)+\left(\begin{smallmatrix}
                  0 \\ Q
              \end{smallmatrix}\right)\),
        \item More general regions can be formed by ``glueing'' together simpler regions.
    \end{enumerate}
\end{proof}




\subsection{Simply connected regions}


{Simply connected regions}


\begin{definition}[simply connected]
    The set \(S\subset \bR^n\) is said to be \emph{simply connected} if every closed path \(\aalpha(t)\) can be continuously shrunk to a point.
\end{definition}



% \begin{figure}
%     \begin{tikzpicture}
%         \fill[paleBlue] plot [smooth cycle, tension=1] coordinates {(0,0) (1,3) (2,1) (3,3) (4,1)};
%         \draw[->,thick] (1,2) -- (1,.8) -- (2.5,.8) -- (2.5,1.5) -- (3.2,1.5) -- (3.2,.5) -- (.5,.5) -- (.5,2) -- (1,2);
%         \node at (3.4,2.5){\(S\)};
%         \node at (1.5,0.3){\(\aalpha\)};
%     \end{tikzpicture}
%     \caption{Simply connected.}
% \end{figure}



% \begin{figure}
%     \begin{tikzpicture}
%         \fill[paleBlue] plot [smooth cycle, tension=1] coordinates {(0,0) (1,3) (3,3) (4,1)};
%         \fill[white] plot [smooth cycle, tension=1] coordinates {(1,1) (2,1) (2.5,2)};
%         \draw[->,thick] (1.2,2) -- (.7,2) -- (.7,.2) -- (2.3,.2) -- (2.3,1) -- (3.2,1) -- (3.2,2.8) -- (1.2,2.8) -- (1.2,2);
%         \node at (1.5,0.4){\(\aalpha\)};
%         \node at (1.8,2.3){\(S\)};
%     \end{tikzpicture}
%     \caption{Not simply connected.}
% \end{figure}






\subsection{Application to conservative vector fields}




\begin{theorem}[conservative vector fields on simply connected regions]
    Let \(S\) be a simply connected region and suppose that \(\ff = \left(\begin{smallmatrix}
            P \\ Q
        \end{smallmatrix}\right)\)
    is a vector field, continuously differentiable on \(S\).
    Then \(\ff\) is conservative if and only if \(\tfrac{\partial Q}{\partial x} = \tfrac{\partial P}{\partial y}\).
\end{theorem}

\begin{proof}
    \begin{enumerate}
        \item We already proved that \(\tfrac{\partial Q}{\partial x} = \tfrac{\partial P}{\partial y}\) whenever \(\ff\) is conservative;
        \item Now suppose that  \(\tfrac{\partial Q}{\partial x} = \tfrac{\partial P}{\partial y}\) and consider any closed path \(\aalpha\) in \(S\),
        \item By Green's theorem \(\int_{C} \ff \cdot d\aalpha = \iint_{S} \left(\tfrac{\partial Q}{\partial x} - \tfrac{\partial P}{\partial y}\right) \ dxdy = 0\),
        \item By the conservative vector field theorem this implies that \(\ff\) is conservative.
    \end{enumerate}
\end{proof}

{Remarks:}
\begin{itemize}
    \item     Invariance of a line integral under deformation of a path.
    \item Multiply connected regions.
\end{itemize}




\section{Change of variables}

\subsection{Jacobian determinant}


{Jacobian determinant}

<>{Recall 1D case:}
If \(g : [a,b] \to [g(a),g(b)]\) is onto with continuous derivative and \(f\) is continuous then
\[
    \int_{g(a)}^{g(b)}    f(x) \ dx = \int_{a}^{b} f(g(u)) \ g'(u) \ du.
\]

<>{Two dimensional change of coordinates}<>
We consider maps \((u,v) \mapsto (X(u,v),Y(u,v))\) mapping  \(T\subset \bR^2\) to  \(S\subset \bR^2\).
% \begin{itemize}
%     \item Is the map differentiable?
%     \item Is the map one-to-one?
% \end{itemize}






\subsection{Change of variables}


{Change of variables}

{Change of variables formula}
Suppose that \((u,v) \mapsto (X(u,v),Y(u,v))\) which maps \(T\) to \(S\) is one-to-one with \(X\), \(Y\) continuously differentiable. Then
\[
    \iint_{S} f(x,y) \ dxdy = \iint_{T} f(X(u,v),Y(u,v)) \ \abs{J(u,v)} \ dudv.
\]
where
\(J(u,v):= \begin{pmatrix}
    \partial_u X & \partial_u Y \\ \partial_v X & \partial_v Y
\end{pmatrix}\) is the Jacobian matrix.


    {Remark}
The Jacobian is scaling of area: \(\displaystyle\iint_{S} \ dx dy = \displaystyle\iint_{T}  \abs{J(u,v)} \ dudv \).




\subsection{Polar coordinates}


{Polar coordinates}

Coordinate mapping:
\begin{itemize}
    \item  \(x = r \cos \theta\)
    \item  \(y = r \sin \theta\)
\end{itemize}

Jacobian determinant:
\[
    \abs{J(r,\theta)}
    =
    \abs{\begin{pmatrix}
            \partial_u X & \partial_u Y \\ \partial_v X & \partial_v Y
        \end{pmatrix}}
    =
    \abs{\begin{pmatrix}
            \cos \theta & \sin \theta \\ -r\sin \theta & r\cos \theta
        \end{pmatrix}}
    = r ( \cos^2\theta + \sin^2 \theta) = r.
\]

Change of coordinates:
\[
    \iint_{S} f(x,y) \ dxdy = \iint_{T}  r  \ f(r\cos \theta, r\sin \theta)  \ drd\theta.
\]





{Change of variables for linear transformations}

Coordinate mapping:
\begin{itemize}
    \item \(x = Au + Bv\)
    \item \(y = Cu + Dv\)
\end{itemize}


Jacobian determinant:
\[
    \abs{J(u,v)}
    =
    \abs{\begin{pmatrix}
            \partial_u X & \partial_u Y \\ \partial_v X & \partial_v Y
        \end{pmatrix}}
    =
    \abs{\begin{pmatrix}
            A & B \\ C & D
        \end{pmatrix}}
    = \abs{AD - BC}.
\]

Change of coordinates:
\[
    \iint_{S} f(x,y) \ dxdy = \abs{AD - BC} \iint_{T}   f(Au + Bv,Cu + Dv)  \ dudv.
\]



% 
%     {Proof of change of variables formula in a special case}

%     We suppose that \(S\) is a rectangle and \(f\) is identically \(1\). I.e., we prove that
%     \[
%       \iint_{R} \ dx dy = \iint_{R'} \abs{J(u,v)} \ duduv  
%     \]
%     where \(R\) is a rectangle in the \(xy\)-plane and \(R'\) is its image in the \(uv\)-plane.


%     \[
%         u = U(x,y), \quad v=V(x,y)
%     \]

%     \[
%         x = X(u,v), \quad y = Y(u,v)
%     \]

%     We use Green's theorem. 



% 


{Extension to higher dimensions}


% Integrability, evaluation by repeated one-dimensional integration, change of variables.

\begin{itemize}
    \item Integrability defined using step functions for \(\iiint_{S} f(x,y,z) \ dxdydz\)
    \item Repeated one dimensional integration: If \(V = \{(x,y,z) : (x,y) \in S, \psi_1(x,y) \leq z \leq \psi_2(x,y)\}\),
          \[
              \iiint_{S} f(x,y,z) \ dxdydz = \iint_{S} \left[\int_{\psi_1(x,y)}^{\psi_2(x,y)} f(x,y,z)\ dz\right] \ dxdy
          \]
    \item Change of variables \((u,v,w) \mapsto (X(u,v,w),Y(u,v,w),Z(u,v,w))\)
          \[
              \iiint_{S} f(x,y,z) \ dxdydz = \iiint_{T} f(X(u,v,w),Y(u,v,w),Z(u,v,w)) \ \abs{J(u,v,w)} \ dudvdw.
          \]
          where
          \(J(u,v):= \begin{pmatrix}
              \partial_u X & \partial_u Y & \partial_u Z \\ \partial_v X & \partial_v Y & \partial_v Z \\ \partial_w X & \partial_w Y & \partial_w Z
          \end{pmatrix}\)
\end{itemize}



\subsection{Cylindrical coordinates}


{Cylindrical coordinates}



Coordinate mapping (require \(r>0\), \(0\leq \theta \leq 2\pi\)):
\begin{itemize}
    \item \(x = r \cos \theta\)
    \item \(y = r \sin \theta\)
    \item \(z = z\)
\end{itemize}



Jacobian determinant:
\[
    \abs{J(r,\theta,z)}
    =
    \abs{\begin{pmatrix}
            \cos \theta    & \sin \theta   & 0 \\
            -r \sin \theta & r \cos \theta & 0 \\
            0              & 0             & 1
        \end{pmatrix}}
    =
    \abs{r (\cos^2 \theta + \sin^2 \theta)}
    = r.
\]

Change of coordinates:
\[
    \iiint_{S} f(x,y,z) \ dxdydz =  \iiint_{T} r \ F(r,\theta,z)   \ dr d\theta dz.
\]
where \( F(r,\theta,z) = f(r \cos \theta, r \sin \theta,  z) \).




\subsection{Spherical coordinates}


{Spherical coordinates}

Coordinate mapping (require \(\rho>0\), \(0\leq \theta \leq 2\pi\), \(0\leq \varphi <\pi\)):
\begin{itemize}
    \item \(x = \rho \cos \theta \sin \varphi\)
    \item \(y = \rho \sin \theta \sin \varphi\)
    \item \(z = \rho \cos \varphi\)
\end{itemize}



Jacobian determinant:
\[
    \abs{J(\rho,\theta,\varphi)}
    =
    \abs{\begin{pmatrix}
            \cos \theta \sin \varphi       & \sin \theta \sin \varphi      & \cos \varphi        \\
            -\rho \sin \theta \sin \varphi & \rho \cos \theta \sin \varphi & 0                   \\
            \rho \cos \theta \cos \varphi  & \rho \sin \theta \cos \varphi & - \rho \sin \varphi
        \end{pmatrix}}
    =
    \abs{- \rho^2 \sin \varphi}
    = \rho^2 \sin \varphi.
\]

Change of coordinates:
\[
    \iiint_{S} f(x,y,z) \ dxdydz =  \iiint_{T}  F(\rho,\theta,\varphi) \rho^2 \sin \varphi  \ d\rho d\theta d\varphi.
\]
where \(F(\rho,\theta,\varphi) = f(\rho \cos \theta \sin \varphi, \rho \sin \theta \sin \varphi,   \rho \cos \varphi  ) \).