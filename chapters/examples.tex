% !TeX root = ../main.tex
\chapter{Worked examples}

\lettrine{T}{his} chapter contains various examples from the different parts of the course and the full calculations.
Some of the examples come from the exercises, some from past exams, some just because they are relevant to the material.

\section{Sequences and series of functions}

\section{Differential calculus in higher dimension}

\begin{task}
    Let $\mathbf{f}:\mathbb{R}^2\to\mathbb{R}^2$, $\mathbf{g}:\mathbb{R}^3\to\mathbb{R}^2$ be defined as
    \[
        \begin{aligned}
            \mathbf{f}(x,y)   & = (e^{x+2y}, \sin(y+2x)) \\
            \mathbf{g}(u,v,w) & = (u+2v^2+3w^3,2v-u^2).
        \end{aligned}
    \]
    Let $\mathbf{h} =\mathbf{f} \circ \mathbf{g} : \bR^3 \to \bR^2$ and calculate \(D\mathbf{h}(1,-1,1)\).
\end{task}

\begin{solution}
    We first calculate \(\mathbf{g}(1,-1,1) = (6,-3)\) and
    \[
        \begin{aligned}
            D\mathbf{f}(x,y)   & =
            \begin{pmatrix}
                e^{x+2y}     & 2 e^{x+2y} \\
                2 \cos(y+2x) & \cos(y+2x)
            \end{pmatrix}, \\
            D\mathbf{g}(u,v,w) & =
            \begin{pmatrix}
                1   & 4v & 9w^2 \\
                -2u & 2  & 0
            \end{pmatrix}.
        \end{aligned}
    \]
    Consequently
    \[
        \begin{aligned}
            D\mathbf{f}(6,-3)   & =
            \begin{pmatrix}
                1        & 2      \\
                2 \cos 9 & \cos 9
            \end{pmatrix}, \\
            D\mathbf{g}(1,-1,1) & =
            \begin{pmatrix}
                1  & -4 & 9 \\
                -2 & 2  & 0
            \end{pmatrix}.
        \end{aligned}
    \]
    By the chain rule for Jacobian matrices (Theorem~\ref{thm:jacobian-chain}),
    \( D\mathbf{h}(1,-1,1) = D\mathbf{f}(6,-3)  D\mathbf{g}(1,-1,1) \)
    and so, multiplying the matrices, we obtain
    $$D\mathbf{h}(1,-1,1) =
        \begin{pmatrix}
            -3 & 0        & 9        \\
            0  & -6\cos 9 & 18\cos 9
        \end{pmatrix}.
    $$
\end{solution}


\section{Applications of the differential calculus}

% \begin{task}
%     Find the extrema of \(f(x,y) = xy\) subject to the constraint \(g(x,y) = x+y-1 =0\).
% \end{task}

% \begin{proof}[Solution]
%     We start by calculating that
%     \[
%         \nabla f(x,y) = \left(\begin{smallmatrix}
%                 y\\ x
%             \end{smallmatrix}\right),
%         \quad
%         \nabla g(x,y) = \left(\begin{smallmatrix}
%                 1\\ 1
%             \end{smallmatrix}\right).
%     \]
%     According to the Lagrange multiplier method there is \(\lambda\in \bR\) such that \(\nabla f(x,y) = \lambda \nabla g(x,y)\) at the extremum point \((x,y)\).
%     To proceed we must solve the system of equations (3 equations and 3 unknowns),
%     \[
%         \left(\begin{smallmatrix}
%                 y\\ x
%             \end{smallmatrix}\right)
%         = \lambda \left(\begin{smallmatrix}
%                 1\\ 1
%             \end{smallmatrix}\right),
%         \quad g(x,y) =0;
%     \]
%     That is,
%     \( x = \lambda, \quad
%     y = \lambda, \quad
%     x+y = 1
%     \).
%     This has the solution \((x,y) = (\frac{1}{2},\frac{1}{2})\), \(f(\frac{1}{2},\frac{1}{2})= \frac{1}{4}\).
% \end{proof}



% \begin{task}
%     Find the points closest to the origin on the set defined by the intersection of the two surfaces
%     \[
%         x^2 - xy + y^2 - z^2 = 1
%         \quad \text{and} \quad
%         x^2 + y^2 = 1.
%     \]
% \end{task}

% \begin{proof}[Solution]
%     For convenience we let
%     \[
%         \begin{aligned}
%             f(x,y,z)   & = x^2 + y^2 + z^2,          \\
%             g_1(x,y,z) & = x^2 - xy + y^2 - z^2 - 1, \\
%             g_2(x,y,z) & = x^2 + y^2 - 1.
%         \end{aligned}
%     \]
%     In the language of the Lagrange multiplier method, we are finding the extrema of \(f\) subject to the constraints \(g_1=0\) and \(g_2 = 0\).
%     Applying the method leads us to the a system of 5 equations and 5 unknowns,
%     \[
%         \nabla f  = \lambda_1 \nabla g_1  + \lambda_2 \nabla g_2,
%         \quad
%         g_1 = 0,
%         \quad
%         g_2 = 0.
%     \]
%     We proceed to solve this system of equations in order to obtain a set of points which are the points where the extrema occur.
%     We calculate that the gradients are
%     \[
%         \nabla f (x,y,z) = \begin{pmatrix}
%             2 x \\ 2y \\ 2z
%         \end{pmatrix},
%         \quad
%         \nabla g_1(x,y,z) = \begin{pmatrix}
%             2x-y \\ 2y - x \\ -2z
%         \end{pmatrix},
%         \quad
%         \nabla g_2(x,y,z) = \begin{pmatrix}
%             2x \\ 2y \\ 0
%         \end{pmatrix}.
%     \]
%     Consequently the 5 equations are
%     \[
%         \begin{aligned}
%             2x & = \lambda_1 ( 2x -y ) + \lambda_2(2x) \\
%             2y & = \lambda_1 ( 2y - x) + \lambda_2(2y) \\
%             2z & = \lambda_1(-2z),
%         \end{aligned}
%     \]
%     \[
%         x^2 + y^2 = 1,
%         \quad
%         x^2 - xy + y^2 - z^2 =1.
%     \]
%     If we combine the 4\textsuperscript{th} and 5\textsuperscript{th} equations we obtain that 
%         \begin{equation}
%             \label{eq:LagrangeA}
%             xy + z^2 = 0.
%         \end{equation}
%     Consequently
%     \begin{equation}
%         \label{eq:LagrangeB}
%         xy \leq 0.
%     \end{equation}
%     If we multiply the 1\textsuperscript{st} by \(y\), multiply the 2\textsuperscript{nd} by \(x\) and combine we obtain that \(\lambda_1(2x-y)y = \lambda_1(2y - x)x\).
%     This means that
%     \begin{equation}
%         \label{eq:LagrangeC}
%         x^2 = y^2 
%         \quad \text{or} \quad
%         \lambda_1 = 0.
%     \end{equation}
%     For a moment we assume the first case and combined this with the 4\textsuperscript{th} equation.
%     This means that \(x^2 +x^2 = 1\) and so \(x = \pm \frac{1}{\sqrt{2}}\).
%     In general \(x^2 = y^2\) allows that \(y = \pm x\) but \eqref{eq:LagrangeB} means that \(y = -x\). Using \eqref{eq:LagrangeA} to calculate \(z\) we obtain 4 solutions,
%     \[
%         (\tfrac{-1}{\sqrt{2}},\tfrac{1}{\sqrt{2}},\tfrac{-1}{\sqrt{2}}),
%         (\tfrac{-1}{\sqrt{2}},\tfrac{1}{\sqrt{2}},\tfrac{1}{\sqrt{2}}),
%         (\tfrac{1}{\sqrt{2}},\tfrac{-1}{\sqrt{2}},\tfrac{-1}{\sqrt{2}}),
%         (\tfrac{1}{\sqrt{2}},\tfrac{-1}{\sqrt{2}},\tfrac{1}{\sqrt{2}}).
%     \]
%     We check that these really are solutions by substituting into the 5 equations.
%     Now we need to consider the other case \eqref{eq:LagrangeC} which we previously ignored. 
%     Consider the 3\textsuperscript{rd} equation we find that \(z = 0\). 
%     Consequently, by \eqref{eq:LagrangeA}, either \(x=0\) or \(y=0\).
%     This then means that, by the 4\textsuperscript{th} equation that \(y^2=1\) or \(x^2=1\) respectively.
%     Consequently we have obtained another 4 solutions,
%     \[
%         (-1,0,0),
%         (1,0,0),
%         (0,-1,0),
%         (0,1,0).
%     \]
%     Again we check that these really are solutions by substituting into the 5 equations.

%     Calculating the distance of the points to the origin we find that the first set are equally the closest to the origin and the second set are equally the furthest.
% \end{proof}



% \begin{task}
%     Investigate the stationary points of the, scalar field,
%     $$ f(x,y,z) = 2 e^{(x-1)^2}(y^4 - 4yz + 2z^2). $$
% \end{task}

% \begin{solution}
%     First, we compute the gradient $\nabla f$:
%     $$ \nabla f(x,y,z) =  \left(\begin{array}{c} \fbox{a}(x-1)e^{(x-1)^2}(y^4 - 4yz + 2z^2) \\
%                 e^{(x-1)^2}(\fbox{b}y^{\fbox{c}} + \fbox{d}z)  \\
%                 e^{(x-1)^2}(\fbox{e}y + \fbox{f}z)             \\
%             \end{array}\right).$$

%     The equation $\nabla f(x,y,z) = \pmb{0}$ has three solutions. They are $(x,y,z) = (\fbox{g}, \sqrt{\fbox{h}}, \sqrt{\fbox{i}}),
%         (\fbox{j}, -\sqrt{\fbox{k}}, -\sqrt{\fbox{l}})$ and $(1,0,0)$ where

%     Consider the first of them $(\fbox{g}, \sqrt{\fbox{h}}, \sqrt{\fbox{i}})$.

%     At this point, the function $f(x,y,z)$ takes a
%     At this point, the function $f(x,y)$ has a
% \end{solution}